\documentclass{article}
\usepackage{graphicx} % Required for inserting images

\usepackage{semantic} % Inferencia

\usepackage[nounderscore]{syntax} %Gramatica \begin{grammar}

\usepackage{amsmath}

%% Sets page size and margins
\usepackage[a4paper,top=2.5cm,bottom=2.5cm,left=2cm,right=2cm]{geometry}

\usepackage{tikz} % Graficas

% \usepackage{float} % Mantener figuras en una posicion

%% Title
\title{
		\vspace{-0.7in} 	
		\usefont{OT1}{bch}{b}{n}
		
		\begin{minipage}{3cm}
        \vspace{-0.5in} 	
    	\begin{center}
    		%\includegraphics[height=3.2cm]{logo_unam.png}
    	\end{center}
    \end{minipage}\hfill
    \begin{minipage}{10.7cm}
    	\begin{center}

\normalfont \normalsize \textsc{UNIVERSIDAD NACIONAL AUTÓNOMA DE MÉXICO \\ FACULTAD DE CIENCIAS \\ LENGUAJES DE PROGRAMACIÓN } \\
		\huge Semanal 03

    	\end{center}
     
    \end{minipage}\hfill
    \begin{minipage}{3.2cm}
    \vspace{-0.5in} 
    	\begin{center}
    		%\includegraphics[height=3.2cm]{logo_fc.png}
    	\end{center}
    \end{minipage}

\author{Ana Lilia Carballido Camacateco - 315314601 && Miguel Angel Vargas Campos - 423114223 && Oscar Fernando Frias Dominguez - 314255662}
\date{}

}

\begin{document}

\maketitle 

\begin{enumerate}
    \item Dadas las siguientes expresiones en sintaxis concreta de nuestro lenguaje MiniLisp (a) obtener su sintaxis abstracta, (b) evaluarlas usando las reglas de semántica natural y (c) evaluarlas usando las reglas de semántica estructural. Todas las reglas las podrán consultar en la Nota de Clase 6 y la Nota de Clase 7.

    \begin{itemize}
        \item \texttt{(- (+ 20 3) (- -18 (+ 50 20)))}
        \begin{enumerate}
            \item Árbol de Sintaxis Abstracta
            \begin{figure}[!htb]
                    \centering
                    \begin{tikzpicture}[
                    level 1/.style={sibling distance=60mm, level distance=20mm},
                    level 2/.style={sibling distance=40mm, level distance=20mm},
                    edge from parent/.style={draw, -latex}]
                        \node {$Sub$}
                            child{
                                node{$Add$}
                                child{
                                    node{$Num$}
                                    child{ node {$20$}}
                                }
                                child{
                                    node{$Num$}
                                    child{ node {$3$}}
                                }
                            }
                            child{
                                node{$Sub$}
                                child{
                                    node{$Num$}
                                    child{ node {$-18$}}
                                }
                                child{
                                    node{$Add$}
                                    child{
                                        node{$Num$}
                                        child{ node{$50$}}
                                    }
                                    child{
                                        node{$Num$}
                                        child{ node{$20$}}
                                    }
                                }
                            };
                    \end{tikzpicture}
                \end{figure}
            \item Semántica Natural
            \[
                \inference{
                    \inference{
                        \inference{
                            \inference{
                                \inference{
                                    % Puedes poner varias premisas en la inferencia usando &
                                    % \inference{a -> Num(b) & c -> Num(d)}{add(a,b) -> Num(b+d)}
                                    \inference{Num(-18)\Rightarrow Num(-18), Num(70)\Rightarrow Num(70), Num(-88)\Rightarrow Num(-88)}
                                    {Sub((Num(-18)),(Num(70)))\Rightarrow Num(-88)}
                                }
                                {sub((Num(-18))(Add(Num(50)),(Num(20)))\Rightarrow) Num(70)}
                            }
                            {Num(20)\Rightarrow Num(20),Num(3)\Rightarrow Num(3), Num(23) \Rightarrow Num(23)}
                        }
                        {Add(Num(20),Num(3)) \Rightarrow Num(23),Sub((Num(-18)),(Add(Num(50)),(Num(20)))}
                    }{(Sub(Add(Num(20),Num(3))),(Sub(Num(-18)),(Add(Num(50)),(Num(20)))))}
                }
                {Num(111)}
            \]
            \item Semántica Estructural
            \[
            \inference{
                \inference{
                    \inference{
                        \inference{
                            \inference{
                                \inference{Sub(Num(-18),Num(70))\rightarrow Num(-88)}
                                {Add(Num(50),Num(20))\rightarrow Num(70)}
                            }
                            {Sub(Num(-18),Add(Num(50),Num(20)))}
                        }
                        {Sub(Num(23),Sub(Num(-18))),Add(Num(50),Num(20))}
                    }
                    {Add(Num(20),Num(3))\rightarrow Num(23)}
                }
                {Sub(Add(Num(20),Num(3)),Sub(Num(-18),Add(Num(50),Num(20))))}
            }
            {Num(111)}
            \]
        \end{enumerate}
        
        \item \texttt{(not (+ 1 (- 3 (+ -8 1))))}

        \begin{enumerate}
            \item Árbol de Sintaxis Abstracta

                \begin{figure}[!htb]
                    \centering
                    \begin{tikzpicture}
                        \node {$Not$}
                        child{
                            node {$Add$}
                            child{
                                node {$Sub$}
                                child {
                                    node {$Num$}
                                    child {
                                        node{$3$}
                                    }
                                }
                                child {
                                    node{$Add$}
                                    child{
                                        node{$Num$}
                                        child{
                                            node {$-8$}
                                        }
                                    }
                                    child{
                                        node {$Num$}
                                        child {
                                            node {$1$}
                                        }
                                    }
                                }
                            }
                        };
                    \end{tikzpicture}
                \end{figure}
            
            \item Semántica Natural

                \[
                    \inference{Num(-8 \Rightarrow Num(-8) &
                    Num(1) \Rightarrow Num(1)}{
                        \inference{Add(Num(-8), Num(1)) \Rightarrow Num(-7)}
                            {\inference{Sub(Num(3), Add(Num(-8), Num(-1))) \Rightarrow Num(-4)}
                                {\inference{Add(Num(1), Sub(Num(3), Add(Num(-8), Num(1)))) \Rightarrow Num(-2) }{Not(+1(-3(+ -8  1))) \Rightarrow Bool(False) }}}
                    }   
                
                \]
            
            \item Semántica Estructural
                \[ Not(Add(Num(1), Sub(Num(3), Add(Num(-8), Num(1))))) \]
                \[ \rightarrow Not(Add(Num(1), Sub(Num(3), Num(-7)))) \]
                \[ \rightarrow Not(Add(Num(1), Num(-3))) \]
                \[ \rightarrow Not(-2) \]
                \[ \rightarrow Bool(False) \]
                

        \end{enumerate}
        
        \item \texttt{(not (not (+ 3 5)))}

        \begin{enumerate}
            \item Árbol de Sintaxis Abstracta
            
                \begin{figure}[!htb]
                    \centering
                    \begin{tikzpicture}
                        \node {$Not$}
                        child{
                            node {$Not$}
                            child{
                                node {$Add$}
                                child{
                                    node {$Num$}
                                    child { node {$3$} }
                                }
                                child{
                                    node {$Num$}
                                    child { node {$5$} }
                                }
                            }
                        };
                    \end{tikzpicture}
                \end{figure}
                
            \item Semántica Natural
            
                \[ 
                    \inference{
                        \inference{
                            \inference{
                                Num(3) \Rightarrow Num(3) &
                                Num(5) \Rightarrow Num(5)
                            }{
                                Add(Num(3),Num(5)) \Rightarrow Num(8)
                            }
                        }{
                            Not(Add(Num(3),Num(5))) \Rightarrow Bool(False)
                        }
                    }{
                        Not(Not(Add(Num(3),Num(5)))) \Rightarrow Bool(True)
                    }
                \]

            \item Semántica Estructural
                \[ Not(Not(Add(Num(3),Num(5)))) \]
                \[ \rightarrow Not(Not(Num(8))) \]
                \[ \rightarrow Not(Bool(False)) \]
                \[ \rightarrow Bool(True) \]
        \end{enumerate}
    \end{itemize}

    \item Como segundo ejercicio deberán extener la batería de operaciones de MiniLisp, para ello deberán (a) dar la gramática libre de contexto modificada (en notación EBNF) añadiendo las nuevas construcciones del lenguaje, (b) modificar las reglas de sintaxis abstracta para considerar los nuevos constructores y finalmente (c) extender las reglas de semántica natural y estructural. En los tres casos, deberás usar la notación formal que vimos en clase.

    \begin{itemize}
        \item Especificar un nuevo constructor $*$ para la multiplicación binaria de expresiones aritméticas. Por ejemplo:

        \begin{verbatim}
            > (* 20 2)
            40
        \end{verbatim}

        \begin{enumerate}
            \item Gramática Libre de Contexto
            A la gramatica previa le añadimos:
            \\
            \begin{grammar}
                <Expr> ::= (mult <Expr><Expr>)
            \end{grammar}
            \item Sintaxis Abstracta
                \[
                    \inference{e,e' \quad \textbf{ASA}}{Mult(e,e') \quad \textbf{ASA}}[(1)]
                \]
            \item Semántica Natural y Estructural
            \begin{enumerate}
                \item Semántica Natural
                \[
                \inference{i\Rightarrow Num(i') & j\Rightarrow Num(j')}{mult(i,j)\Rightarrow Num(i'*j')}
                \]
                \item Semántica Estructural\\
                Caso 1: 
                \[ \inference{i\rightarrow i'}{Mult(i,j)\rightarrow Mult(i',j)}\]
                Caso 2:
                \[ \inference{j\rightarrow j'}{Mult(Num(i),j)\rightarrow Mult(Num(i),j')}\]
                Caso 3:
                \[ \inference{}{Mult(Num(i),Num(j))\rightarrow Num(i*j)}\]
            \end{enumerate}
        \end{enumerate}

        \item Especificar un nuevo constructor $/$ para la división binaria de expresiones aritméticas. Consideren que no se pueden realizar divisiones entre cero. Por ejemplo:

        \begin{verbatim}
            > (/ 20 2)
            10
            > (/ 10 0)
            error: División entre cero
        \end{verbatim}

        \begin{enumerate}
            \item Gramática Libre de Contexto
            \\
            A la gramática previa le añadimos:
            \\
            \begin{grammar}
                <Expr> ::= (Div <Expr><Expr>)
            \end{grammar}
            \item Sintaxis Abstracta
                \[
                    \inference{e,e' \quad \textbf{ASA}}{Div(e,e') \quad \textbf{ASA}}[(1)]
                \]
            \item Semántica Natural y Estructural
            \begin{enumerate}
                \item Semántica Natural
                \[ \inference{i\Rightarrow Num(i') & j\Rightarrow Num(0)}{Div(i,j)\Rightarrow \text{Error: división entre cero}}\]
                \[ \inference{i\Rightarrow Num(i') & j\Rightarrow Num(j')}{Div(i,j) \Rightarrow Num(i'/j')}\]
                \item Semántica Estructural\\
                Caso 1:
                \[ \inference{i \rightarrow i'}{Div(i,Num(0))\Rightarrow \textbf{Error: División entre cero}}\]
                Caso 2:
                \[ \inference{i\rightarrow i'}{Div(i,j)\rightarrow Div(i',j)}\]
                Caso 3:
                \[ \inference{j\rightarrow j'}{Div(Num(i),j)\rightarrow Div(Num(i),j')}\]
                Caso 4:
                \[ \inference{}{Div(Num(i),Num(j))\rightarrow Num(i/j)}\]
            \end{enumerate}
        \end{enumerate}

        \item Especificar un nuevo constructor $add1$ que dada una expresión, incrementa en uno su valor. Por ejemplo: 

        \begin{verbatim}
            > (add1 10)
            11
        \end{verbatim}

        \begin{enumerate}
            \item Gramática Libre de Contexto
            \\
            A la gramática previa le añadimos:
            \\
            \begin{grammar}
                <Expr> ::= (Add1 <Expr>)
            \end{grammar}
            \item Sintaxis Abstracta
                \[
                \inference{e \quad \textbf{ASA}}{Add1(e) \quad \textbf{ASA}}[(1)]
                \]
            \item Semántica Natural y Estructural
            \begin{enumerate}
                \item Semántica Natural
                    \[
                        \inference{n \Rightarrow Num(m)}{Add1(n) \Rightarrow Num(m + 1)}[(1)]
                    \]
                \item Semántica Estructural

                    \[ \inference{n\rightarrow m}{Add1(n) 
                        \rightarrow Add1(m)}[(1)] \]
                    \[ \inference{}{Add1(Num(n)) \rightarrow Num(n + 1)}[(2)] \]
                
            \end{enumerate}
        \end{enumerate}

        \item Especificar un nuevo constructor $sub1$ que dada una expresión, decrementa en uno su valor. Por ejemplo:

        \begin{verbatim}
            > (sub1 10)
            9
        \end{verbatim}

        \begin{enumerate}
            \item Gramática Libre de Contexto
            \\ \\
            A la gramatica previa le añadimos:
            \\
            \begin{grammar}
                <Expr> ::= (sub1 <Expr>)
            \end{grammar}
            
            \item Sintaxis Abstracta
                \[
                \inference{e \quad \textbf{ASA}}{Sub1(e) \quad \textbf{ASA}}[(1)]
                \]
            \item Semántica Natural y Estructural
            \begin{enumerate}
                \item Semántica Natural
                    \[
                        \inference{n \Rightarrow Num(n')}{Sub1(n) \Rightarrow Num(n' - 1)}[(1)]
                    \]

                    %\[
                    %    \inference{n \Rightarrow Bool(m)}{sub1(n) \Rightarrow Not(Bool(m))}[(2)]
                    %\]
                \item Semántica Estructural
                    \[ \inference{n\rightarrow n'}{Sub1(n) 
\rightarrow Sub1(n')}[(1)] \]
                    \[ \inference{}{Sub1(Num(n)) \rightarrow Num(n - 1)}[(2)] \]
            \end{enumerate}
        \end{enumerate}

        \item Especificar un nuevo constructor $sqrt$ que dada una expresión, obtiene la raíz cuadrada de dicha expresión. Consideren que no se pueden calcular raíces cuadradas de números negativos. Por ejemplo:

        \begin{verbatim}
            > (sqrt 81)
            9
            > (sqrt -2)
            error: Raíz negativa
        \end{verbatim}

        \begin{enumerate}
            \item Gramática Libre de Contexto
            \\ \\
            A la gramatica previa le añadimos:
            \\
            \begin{grammar}
                <Expr> ::= (sqrt <Expr>)
            \end{grammar}
            \item Sintaxis Abstracta
                \[
                    \inference{e \quad \textbf{ASA}}{Sqrt(e) \quad \textbf{ASA}}[(1)]
                \]
            \item Semántica Natural y Estructural
            \begin{enumerate}
                \item Semántica Natural

                    \[
                        \inference{n \Rightarrow Num(n') \text{ con } n' < 0}{Sqrt(n) \Rightarrow \text{error: Raíz negativa}}[(1)]
                    \]

                    \[
                        \inference{n \Rightarrow Num(n')}{Sqrt(n) \Rightarrow Num(\sqrt{n'})}[(2)]
                    \]
                
                \item Semántica Estructural

                    \[
                        \inference{n \rightarrow n'}{Sqrt(n) \rightarrow Sqrt(n')}[(1)]
                    \]

                    \[
                        \inference{n < 0}{Sqrt(Num(n)) \rightarrow \text{error: Raíz negativa}}[(2)]
                    \]

                    \[
                        \inference{}{Sqrt(Num(n)) \rightarrow Num(\sqrt{n})}[(3)]
                    \]
                    
            \end{enumerate}
        \end{enumerate}
         
    \end{itemize}

\end{enumerate}

\end{document}
